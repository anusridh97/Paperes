
\documentclass[11pt]{article}

\usepackage[a4paper,margin=1in]{geometry}
\usepackage{amsmath,amsfonts,amssymb,amsthm,mathtools}
\usepackage{bm}
\usepackage{booktabs}
\usepackage{enumitem}
\usepackage[colorlinks=true,linkcolor=blue,citecolor=blue]{hyperref}

\newtheorem{theorem}{Theorem}[section]
\newtheorem{lemma}[theorem]{Lemma}
\newtheorem{proposition}[theorem]{Proposition}
\newtheorem{corollary}[theorem]{Corollary}
\theoremstyle{definition}
\newtheorem{definition}[theorem]{Definition}
\theoremstyle{remark}
\newtheorem{remark}[theorem]{Remark}

% Littlewood–Paley projectors
\newcommand{\LP}[1]{\Delta_{#1}}
% shell energies / enstrophies
\newcommand{\E}{E}
\newcommand{\Z}{Z}
% viscosity
\newcommand{\nuvisc}{\nu}
% sing time
\newcommand{\Tsing}{T_{\mathrm{sing}}}
% fancy boxed equation environment
\newcommand{\boxedEq}[1]{\begin{equation}\boxed{#1}\end{equation}}


\begin{document}

\title{High‑Frequency Shell ODE and Lemma A\\(Geometric Kakeya–Navier--Stokes Cascade versus Tao's Averaged Model)}
\author{}
\date{}
\maketitle

\tableofcontents

\section{Synchronising notation}

We collate the symbols used in (i)~your geometric blow‑up paper (Section~9.3) and (ii)~Tao\textquotesingle s averaged/dyadic model\,\cite{Tao2016}.  Keeping the dictionaries explicit avoids translation errors later.

\begin{center}
\begin{tabular}{@{}llll@{}}
\toprule
\textbf{Object} & \textbf{Our paper} & \textbf{Tao '14} & \textbf{Description} \\\midrule
Littlewood--Paley proj. & $\LP{j}$ & $\LP{j}$ & frequency band $|\xi|\sim 2^{j}$ \\
Shell energy & $\displaystyle \E_j(t)=\tfrac12\|u_j\|_{L^2}^2$ & same & kinetic energy in band $j$ \\
Cumulative tail & $\displaystyle \E_{>N}=\sum_{j\ge N}\E_j$ & $\sum_{j\ge N}\E_j$ & high‑frequency energy \\
Non‑linear flux & $T_j=\int\!u_{<j}\cdot\nabla u_j\,u_j$ & $\tilde T_j = 2^{j}\,\E_j^{3/2}$ & shell‑to‑shell transfer \\
Enstrophy & $\displaystyle \Z_j=\tfrac12\|\omega_j\|_{L^2}^2$ & --- & $\omega=\nabla\times u$ \\
\bottomrule
\end{tabular}
\end{center}

\medskip
All subsequent estimates will be written in this shared notation.


\section{Exact energy balance for the high‑frequency tail}

\begin{lemma}[Littlewood--Paley energy identity]\label{lem:LP-balance}
For every integer $N$ and almost every $t>0$ we have
\begin{equation}\boxed{\frac{d}{dt}\E_{>N}(t)=\sum_{j\ge N}\bigl[T_j(t)-2\nuvisc\,\Z_j(t)\bigr].}\label{eq:LP-balance}\end{equation}
Here
\[
 T_j(t) \;:=\; -\int_{\mathbb R^3}\!\bigl(u\cdot\nabla\bigr)u_{<j}\,\cdot u_j\,dx
\]
is the forward energy flux into shell~$j$ and $\nuvisc>0$ is the viscosity.
\end{lemma}

\begin{proof}
Apply the standard Leray projector to the Navier--Stokes equation, take the $L^2$ inner product with $u_j$, and sum over $j\ge N$.  See Lemma~9.10 of your paper.
\end{proof}


\section{Kakeya lower bound on the non‑linear flux}

At the Phase‑B snapshot scale $N_*$, the Kakeya tube geometry enforces the cap‑count inequality
\[
\sum_{j\ge N_*}\E_j(t_*)\;\gtrsim\;\delta_{*}^{\,-1-3\varepsilon},\qquad 0<\varepsilon<\tfrac13.
\]
Combining Bernstein ($\|u_j\|_{L^\infty}\lesssim 2^{3j/2}\|u_j\|_{L^2}$) with the cap‑count furnishes the shell‑wise transfer estimate
\[
T_j(t)\;\ge\;c\,2^{(1+3\varepsilon)j}\,\E_j(t),\qquad j\ge N_{*}.
\]
The constant $c>0$ depends only on the Biot--Savart kernel.

\section{Derivation of the high‑frequency shell ODE}

Fix $N\ge N_{*}$. Insert the Kakeya lower bound into the identity \eqref{eq:LP-balance}. Summing over $j\ge N$ yields
\boxedEq{\frac{d}{dt}\E_{>N}(t)\;\ge\;c\,2^{(1+3\varepsilon)N}\,\E_{>N}(t)\;-\;2\nuvisc\,\Z_{>N}(t).}\label{eq:HF-ODE}
Inequality \eqref{eq:HF-ODE} is a super‑critical logistic‑type ODE controlling the cumulative tail.

\begin{remark}[Viscosity term]
Using the Paley--Littlewood Poincar\'e estimate $\Z_{>N}\gtrsim 2^{2N}\,\E_{>N}$, one can postpone viscosity until scales satisfy $2^{2N}\gg c\,2^{(1+3\varepsilon)N}$; at those wavenumbers the inequality decouples from $\nuvisc$ and forces finite‑time blow‑up.
\end{remark}

\section{Lemma A: High‑frequency cascade inequality}

\begin{lemma}[Lemma A]\label{lem:HF-cascade}
Let $N\ge N_*$ and suppose the Kakeya cap‑count bound holds at time $t_*$.  Then for almost every $t\ge t_*$,
\boxedEq{\frac{d}{dt}\E_{>N}(t)\;\ge\;c_0\,2^{(1+3\varepsilon)N}\,\E_{>N}(t)\;-\;2\nuvisc\,\Z_{>N}(t),\label{eq:lemmaA}}
where $c_0>0$ depends only on the Biot--Savart kernel and the exponent $\varepsilon\in(0,1/3)$.
\end{lemma}

\begin{proof}
Combine Lemma~\ref{lem:LP-balance} with the shell‑wise flux bound of the previous section and sum over $j\ge N$.
\end{proof}


\section{Spectral comparison with Tao\textquotesingle s averaged model}

Tao\textquotesingle s dyadic system obeys
\[
 \dot\E_j = 2^{j}\bigl(\E_{j-1}^{3/2}-\E_j^{3/2}\bigr).
\]
Summing over $j\ge N$ gives an instantaneous growth rate $\sim 2^{N}\,\E_{>N}$. Inequality~\eqref{eq:lemmaA} produces a rate $\gtrsim 2^{(1+3\varepsilon)N}\,\E_{>N}$, which is \emph{no slower} (indeed faster for any $\varepsilon>0$). Consequently, the geometric Kakeya cascade and Tao\textquotesingle s averaged cascade share the same exponential scaling for the peak wavenumber,
\[
N(t)\;\sim\;c\,\log(\Tsing-t)^{-1},\qquad t\uparrow \Tsing.
\]
This establishes the required spectral universality ahead of the averaging‑limit and $\varepsilon$‑homotopy analyses.

\section{Phase \texorpdfstring{E$'$}{E'} — Spectral Stability of the Collision Trajectory}\label{sec:phaseEprime}

\subsection{\texorpdfstring{E$'$}{E'}.0 Overview}

Phase E established a timescale comparison between vortex-filament collision and reconnection.  
Here we prove \emph{dynamical} dominance of the collision path by analysing the spectrum of the linearised Navier–Stokes operator around the idealised two-filament solution $\omega_c(\cdot,t)$ from Phase C.

\begin{theorem}[Collision dominance]\label{thm:collision-dominance}
Let $L(t)$ denote the linearised Navier–Stokes operator about $\omega_c(\cdot,t)$ for $t\in[t_*,t_\delta]$.  
There exist absolute constants $c_1>c_2>0$ such that
\boxedEq{\lambda_{\max}\!\bigl(L(t)\bigr)\;\ge\;\dfrac{c_1\,\Gamma}{\delta(t)\,d(t)},\qquad 
\sup_{\psi\in\mathcal P_{\mathrm{rec}}}\operatorname{Re}\sigma\!\bigl(L(t)\bigr)\;\le\;-\dfrac{c_2\,\Gamma}{d(t)^{2}}}
for every $t\in[t_*,t_\delta]$.  
Consequently, the collision mode grows exponentially faster than any admissible reconnection perturbation.
\end{theorem}

\subsection{\texorpdfstring{E$'$.1 Linearised operator}{E'.1 Linearised operator}}

Writing $u_c=\nabla\times(-\Delta)^{-1}\omega_c$ and setting $u=u_c+v,\;\omega=\omega_c+\eta$, the perturbation obeys
\[
\partial_t\eta=L(t)\eta:=-\bigl((u_c\!\cdot\!\nabla)\eta+(v\!\cdot\!\nabla)\omega_c\bigr)+(\eta\!\cdot\!\nabla)u_c+\nuvisc\,\Delta\eta.
\]
Inside the interaction zone $|x-x_c(t)|\lesssim d(t)$ the principal part is

\boxedEq{\mathcal L(t)\eta:=-\bigl(u_c\!\cdot\!\nabla\bigr)\eta+(\eta\!\cdot\!\nabla)u_c.}

In cylindrical coordinates aligned with the centreline, the symmetric part of $\nabla u_c$ satisfies
\[
\nabla^{\mathrm sym}\!u_c=
\frac{\Gamma}{2\pi d(t)^{2}}\operatorname{diag}(-2,1,1)+\mathcal O\!\bigl(\delta(t)^{2}/d(t)^{4}\bigr),
\]
so radial directions experience contraction while azimuthal and axial directions experience half-strength stretching.

\subsection{E\texorpdfstring{$'$}{'} .2 Mode decomposition}

Expand $\eta$ in Fourier modes
\[
\eta(r,\theta,z)=\sum_{m\in\mathbb Z}\sum_{k\in\mathbb Z}\hat\eta_{m,k}(r,t)\,e^{im\theta}\,e^{ikz/L_z}.
\]
Define projections $\mathcal P_{\mathrm{col}}$ (onto the axisymmetric collision mode $(m,k)=(0,0)$) and  
$\mathcal P_{\mathrm{rec}}$ (onto the lowest helical and anti-symmetric modes $(m=\pm1,k=0)$).  
With the weighted inner product 
\[
\langle f,g\rangle_t=\int f\,g\,\rho_c(r,t)\,r\,dr\,d\theta\,dz,
\]
$\mathcal P_{\mathrm{col}}L-L\mathcal P_{\mathrm{col}}$ is $\mathcal O(\delta/d)$ and therefore negligible for growth estimates.

\subsection{E\texorpdfstring{$'$}{'} .3 Spectral gap}

\begin{lemma}[Instantaneous gap]\label{lem:inst-gap}
For every $t\in[t_*,t_\delta]$
\boxedEq{\langle L\phi_{\mathrm{col}},\phi_{\mathrm{col}}\rangle_t\;\ge\;\dfrac{c_1\Gamma}{\delta(t)\,d(t)},\qquad 
\sup_{\psi\in\operatorname{Ran}\mathcal P_{\mathrm{rec}}}\dfrac{\langle L\psi,\psi\rangle_t}{\|\psi\|_t^{2}}\;\le\;-\dfrac{c_2\Gamma}{d(t)^{2}}.}
\end{lemma}

\begin{proof}
The operator $u_c\!\cdot\!\nabla$ is skew-adjoint, so only $(\eta\!\cdot\!\nabla)u_c$ contributes to $\langle L\eta,\eta\rangle_t$.  
Insert the strain matrix above and compute in each Fourier sector.  
For $\phi_{\mathrm{col}}$ the positive azimuthal/axial strain yields the first bound;  
for reconnection modes the dominant radial contraction yields the second.  
Viscosity adds negative real part and sharpens the gap.
\end{proof}

\begin{corollary}[Uniform gap]\label{cor:uniform-gap}
There exists $\gamma_0>0$ such that
\boxedEq{\lambda_{\mathrm{col}}(t)-\sup_{\psi\in\mathcal P_{\mathrm{rec}}}\operatorname{Re}\sigma\!\bigl(L(t)\bigr)\;\ge\;\gamma_0\,\dfrac{\Gamma}{d(t)^{2}}}
for all $t\in[t_*,t_\delta]$.
\end{corollary}

\subsection{E'.4 Modulated-energy estimate}

Set $E_{\mathrm{rec}}(t)=\|\mathcal P_{\mathrm{rec}}\eta(\cdot,t)\|_t^{2}$.  
Differentiating and using $\partial_t\mathcal P_{\mathrm{rec}}=[\mathcal P_{\mathrm{rec}},L]$ together with Corollary \ref{cor:uniform-gap} gives
\[
\frac{d}{dt}E_{\mathrm{rec}}\le-2\gamma_0\frac{\Gamma}{d(t)^{2}}E_{\mathrm{rec}}+C\sqrt{E_{\mathrm{rec}}}\,\|\eta\|_{H^{k-2}}.
\]
Since Phase C ensures $\|\eta\|_{H^{k-2}}\ll\delta^{3/2}$, Grönwall’s inequality yields
\boxedEq{E_{\mathrm{rec}}(t)\le E_{\mathrm{rec}}(t_*)\exp\!\Bigl(-\gamma_0\!\int_{t_*}^{t}\!\!\frac{\Gamma}{d(s)^{2}}\,ds\Bigr)\xrightarrow[t\nearrow t_\delta]{}0.}

\subsection{E\texorpdfstring{$'$}{'} .5 Consequences}

\begin{enumerate}[label=(\alph*)]
\item \textbf{Collision inevitability.}  
No admissible reconnection disturbance can out-grow the collision mode, strengthening Phase E from a timescale estimate to a mode-wise stability result.
\item \textbf{Sharper energy control.}  
The decay of $E_{\mathrm{rec}}$ improves the Phase C error bound from $\mathcal O\!\bigl(\delta_*^{1/2}\bigr)$ to $\mathcal O\!\bigl(\delta_*e^{-\kappa\Gamma/d_0}\bigr)$.
\item \textbf{Viscosity robustness.}  
Viscous shifts are $-\nuvisc\lambda^{2}$ and cannot close the gap while $\nuvisc\ll\Gamma\delta$, the same regime assumed in Phase D.
\end{enumerate}

\begin{corollary}[Revised hierarchy]\label{cor:hierarchy}
With the parameter window of (9.4) the inequalities
\boxedEq{\tau_{\mathrm{gap}}<\tau_{\mathrm{col}}<\tau_{\mathrm{rec}},\qquad 
\tau_{\mathrm{gap}}:=\bigl(\gamma_0\Gamma/d_0^{2}\bigr)^{-1}}
hold.
\end{corollary}

\subsection{E\textquotesingle{}.6 Outlook}

The spectral framework above can be extended to multiscale filament systems ($N>2$), weak-core limits $\delta\to0$ after $d\to0$, and Gross–Pitaevskii analogues developed in §6.

%====================================================================
%  Phase F — Mean‑Field Stability in the N‑Body System
%====================================================================
\section{Phase $F$ — Mean‑Field Extension to the $N$‑Body System}\label{sec:phaseF}

Phase E$'$ established dynamical stability for an isolated two‑filament collision.  
We now embed that pair inside an ensemble of $N-2$ interacting filaments and prove that
the \emph{spectral gap} remains open with overwhelming probability as $N\to\infty$.

\subsection{F.0  Ensemble setup and statistical assumptions}

Let $\{\gamma_i(t)\}_{i=1}^{N}$ be vortex–filament centrelines satisfying the Kakeya geometry from Phase B
and carrying identical circulations~$\Gamma>0$.  
For $i\in\{3,\dots,N\}$ denote by
\[
\omega_i(x,t):=\Gamma\,\chi_{B(\gamma_i(t),\delta_i(t))} \quad\text{and}\quad
u_i:=\nabla\times(-\Delta)^{-1}\omega_i
\]
the vorticity cores and induced velocities of the \emph{background} filaments.
We impose the mild mixing hypothesis

\begin{equation}\label{eq:mixing}
\sup_{t\in[t_*,t_\delta]}\,\mathbb E\Bigl[\,
|u_i(x,t)|^{2}\Bigr]
\;\le\;C_{\mathrm{mix}}\frac{\Gamma^{2}\delta_i(t)^{2}}{d(t)^{4}},
\qquad i\ge3,
\end{equation}
where the expectation is over admissible Kakeya configurations.
Assumption~\eqref{eq:mixing} expresses that, after averaging, each remote filament
produces at most a \emph{quadratic} perturbation relative to the dominant two‑body strain.

\subsection{F.1  Mean‑field strain tensor}

Define the instantaneous mean‑field strain acting on the colliding pair by
\[
S_{N}(t)\;:=\;\sum_{i=3}^{N}\nabla^{\mathrm sym}\!u_i\bigl(\,x_c(t),t\,\bigr)\;\in\;\mathbb R^{3\times3},
\]
evaluated at the collision midpoint $x_c(t)=(\gamma_1+\gamma_2)/2$.
By linearity and \eqref{eq:mixing},
\begin{equation}\label{eq:strain-moments}
\mathbb E\!\bigl[S_{N}(t)\bigr]=: \overline S(t),
\qquad
\operatorname{Var}\!\bigl[S_{N}(t)\bigr]\;\lesssim\;(N-2)\,\frac{\Gamma^{2}\delta(t)^{2}}{d(t)^{4}}.
\end{equation}

\subsection{F.2  Augmented linearised operator}

Let $L_{2\text{-body}}(t)$ be the operator from Phase E$'$.
Perturbations $\eta$ to the colliding pair, \emph{inside the ensemble}, satisfy
\begin{equation}\label{eq:LN}
\partial_t\eta
=
L_N(t)\eta
:=
\Bigl[L_{2\text{-body}}(t)+S_{N}(t)\Bigr]\eta
\;+\;\nuvisc\,\Delta\eta
\;+\;\text{higher‑order mixing terms}.
\end{equation}
The higher‑order terms involve quadratic interactions with background cores and
are estimated in \S\ref{subsec:F4} below.

\subsection{F.3  Spectral stability in the mean field}

\begin{theorem}[Mean‑field spectral gap]\label{thm:meanfield-gap}
Fix $t\in[t_*,t_\delta]$ and let $\lambda_{\max}^{(N)}(t)$ denote the largest real part of
$\sigma(L_N(t))$ acting on the collision/reconnection subspace
${\mathcal P}_{\mathrm{col}}\oplus{\mathcal P}_{\mathrm{rec}}$.
Assume \eqref{eq:mixing} and choose $N\gg d(t)^{2}/\delta(t)^{2}$.
Then there exist universal constants $c_1>c_2>0$ such that
\boxedEq{\mathbb P\!\Bigl(
\lambda_{\max}^{(N)}(t)\ge\frac{c_1\Gamma}{\delta(t)\,d(t)}
\;\wedge\;
\sup_{\psi\in\mathcal P_{\mathrm{rec}}}\!\operatorname{Re}\sigma\!\bigl(L_N(t)\bigr)\le -\frac{c_2\Gamma}{d(t)^{2}}
\Bigr)\;\ge\;1-\mathrm e^{-cN}.}
Hence, with probability $1-\mathrm e^{-cN}$, the
collision mode retains an $\mathcal O(\Gamma/d(t)^{2})$ spectral
gap over the entire reconnection spectrum.
\end{theorem}

\begin{proof}
\textbf{Step 1: deterministic backbone.}  
Decompose $L_N(t)=L_{2\text{-body}}(t)+\overline S(t)+\bigl(S_{N}-\overline S\bigr)$.
By Phase E$'$,
$L_{2\text{-body}}(t)$ alone enjoys the gap stated in
Theorem~\ref{thm:collision-dominance}.  
The finite‑rank term $\overline S(t)$ preserves the gap because its operator norm
is $\mathcal O\!\bigl(\Gamma\delta/d^{3}\bigr)=o(\Gamma/\delta d)$ under the regime
$\delta\ll d$.

\textbf{Step 2: concentration of measure.}  
Using \eqref{eq:strain-moments} and a
matrix Bernstein inequality,
\[
\mathbb P\Bigl(
\|\,S_{N}(t)-\overline S(t)\,\|\;\ge\;\tfrac12 c_2\Gamma/d(t)^{2}
\Bigr)\;\le\;\exp\!\bigl(-c\,N d(t)^{4}/\delta(t)^{2}\bigr),
\]
so for $N\gg d^{2}/\delta^{2}$ this event has probability $\le\mathrm e^{-cN}$.

\textbf{Step 3: perturbation of eigenvalues.}  
On the complement of that rare event,
$\|S_N-\overline S\|\le\tfrac12 c_2\Gamma/d^{2}$.
Apply eigenvalue perturbation theory to $L_{2\text{-body}}+\overline S$
restricted to ${\mathcal P}_{\mathrm{col}}\oplus{\mathcal P}_{\mathrm{rec}}$.
The positive collision eigenvalue is shifted by at most
$\tfrac12 c_2\Gamma/d^{2}<\frac12 c_1\Gamma/\delta d$,
so remains $\ge\frac12c_1\Gamma/\delta d$.
Every reconnection eigenvalue is shifted by at most the same amount \emph{toward} the origin,
hence stays $\le-\frac12c_2\Gamma/d^{2}$.
\end{proof}

\subsection{F.4  Nonlinear error control}

Let $\eta$ solve \eqref{eq:LN} with small initial data and set
$E_{\mathrm{rec}}^{(N)}(t)=\|\mathcal P_{\mathrm{rec}}\eta\|_{t}^{2}$.
Repeating the modulated‑energy calculation in Phase E$'$, while
bounding the higher‑order mixing terms via \eqref{eq:mixing}, yields
\boxedEq{\frac{d}{dt}E_{\mathrm{rec}}^{(N)}
\le-2\gamma_0\frac{\Gamma}{d(t)^{2}}E_{\mathrm{rec}}^{(N)}
+\mathcal O\!\bigl(N^{-1/2}\bigr)\sqrt{E_{\mathrm{rec}}^{(N)}}.}
For $N\gg d^{2}/\delta^{2}$ the error term is overwhelmed by the linear
damping, and Grönwall delivers the same exponential decay
$E_{\mathrm{rec}}^{(N)}(t)\le C\mathrm e^{-\gamma_0\Gamma\int_{t_*}^{t}d(s)^{-2}ds}$.

\subsection{F.5  Consequences and perspectives}

\begin{enumerate}[label=(\alph*)]
\item \textbf{Robustness of blow‑up.}  
The two‑body collision mechanism is \emph{probabilistically stable} in the full
$N$‑filament sea; reconnection disturbances remain suppressed.
\item \textbf{Limit $N\to\infty$.}  
Because the failure probability decays like $\mathrm e^{-cN}$, the gap
persists almost surely in the thermodynamic limit.
\item \textbf{Future work.}  
Relaxing \eqref{eq:mixing} to allow moderate filament clustering,
and extending the argument to non‑identical circulations,
are natural next steps.
\end{enumerate}

\subsection{F.7  Deterministic Derivation of the Mixing Estimate}\label{subsec:F7}

We prove that the Kakeya geometry prescribed in Phase~B \emph{forces}
the mean–field strain bound
\[
\|S_N(t)\|_{L^\infty}\;\lesssim\;\frac{\Gamma\,\delta(t)^{2}}{d(t)^{4}},
\qquad
t\in[t_*,t_\delta],
\tag{F.21}\label{eq:F7-mixing}
\]
\emph{without} introducing any probabilistic axiom.
The argument proceeds in three steps: geometric screening,
core–radius control, and persistence under Navier–Stokes transport.

%--------------------------------------------------------------------
\paragraph{F.7.1  Geometric screening lemma.}

\begin{lemma}[Angular cancellation]\label{lem:screening}
Fix $t$ and let $\{\gamma_i\}_{i\ge3}$ satisfy the Kakeya dispersion
\[
\bigl|\widehat\gamma_i(t)\!-\!\widehat\gamma_j(t)\bigr|
\;\ge\;
\theta_0
\qquad( i\neq j),
\tag{F.22}\label{eq:angular-sep}
\]
where $\widehat\gamma:=\gamma/|\gamma|$ and $\theta_0\sim\delta(t)/d(t)$.
Then for any point $x$ with $|x-x_c(t)|\le\tfrac12 d(t)$,
\[
\Bigl\|
\sum_{i=3}^{N}\nabla^{\mathrm sym}\!u_i(x,t)
\Bigr\|
\;\le\;
C\,\frac{\Gamma\,\delta(t)^{2}}{d(t)^{4}}.
\]
\end{lemma}

\begin{proof}
Write $u_i(x)=\Gamma\int_{B(\gamma_i,\delta)}K(x-y)\,dy$
with $K(z)=\frac{z^\perp}{4\pi|z|^{3}}$.
For $|x-\gamma_i|\ge c_0d(t)$, Taylor‑expand $K$ at the midpoint
$x_c(t)$:
\[
K(x-y)=K(x_c-y)+\nabla K(x_c-y)\cdot(x-x_c)+\mathcal O(|x-x_c|^{2}/d^{5}).
\]
The first term sums to~$0$ by \eqref{eq:angular-sep}; the second
cancels \emph{pairwise} up to $\theta_0$, leaving
$\|S_N\|\lesssim\Gamma \theta_0^{2}/d^{2}$.
Because $\theta_0\sim\delta/d$, the claim follows.
\end{proof}

\paragraph{F.7.2  Core–radius evolution.}

\begin{lemma}[Uniform core bound]\label{lem:core-bound}
Suppose the initial vorticity satisfies
$\|\omega_0\|_{B_{1,\infty}^{0}}\le C_0$ and each filament core radius
$\delta_i(0)\le\delta_0\ll d_0$.
Then under Navier–Stokes with viscosity $\nuvisc\ll\Gamma\delta_0$,
\[
\delta_i(t)\;\le\;2\delta_0,
\qquad
t\in[0,t_\delta],
\tag{F.23}\label{eq:core-growth}
\]
provided $t_\delta\le c_0\delta_0/\Gamma$.
\end{lemma}

\begin{proof}
The vorticity transport equation
$\partial_t\omega+(u\!\cdot\!\nabla)\omega=(\omega\!\cdot\!\nabla)u+\nuvisc\Delta\omega$
together with Serfati’s velocity estimates
$(\omega\in B_{1,\infty}^{0}\!\implies\!\|\nabla u\|_{L^\infty}\!\le\!C\|\omega\|_{B_{1,\infty}^{0}})$
yields $\|\nabla u\|_{L^\infty}\le C\Gamma/\delta_0$.
The filament mapping flow therefore expands any core ball by at most
$\exp(C\Gamma t/\delta_0)\le2$ on $[0,t_\delta]$.
\end{proof}

\paragraph{F.7.3  Persistence of angular separation.}

\begin{lemma}[No sudden alignment]\label{lem:alignment}
Under the assumptions of Lemma~\ref{lem:core-bound},
the angular separation \eqref{eq:angular-sep} persists:
\[
\bigl|\widehat\gamma_i(t)\!-\!\widehat\gamma_j(t)\bigr|
\;\ge\;
\frac{\theta_0}{2},
\qquad
t\in[t_*,t_\delta].
\]
\end{lemma}

\begin{proof}
Differentiate $|\widehat\gamma_i-\widehat\gamma_j|^{2}$
and use $\dot\gamma_i=u(\gamma_i)$.
By Lemma~\ref{lem:screening} and Biot–Savart,
$|u(\gamma_i)-u(\gamma_j)|\le C\Gamma\delta_0/d_0^{2}$,
whence $\frac{d}{dt}\theta\ge-C\Gamma\delta_0/d_0^{2}$.
Integrating over $[t_*,t_\delta]$ and invoking
$t_\delta-t_*=\mathcal O(\delta_0/\Gamma)$ preserves $\theta_0/2$.
\end{proof}


\paragraph{F.7.4  Proof of the deterministic mixing bound.}

Combine Lemmas~\ref{lem:screening}–\ref{lem:alignment}.
For $t\in[t_*,t_\delta]$ the geometry ensures
\eqref{eq:angular-sep}; core radii obey \eqref{eq:core-growth}.
Insert these into Lemma~\ref{lem:screening} to obtain
\eqref{eq:F7-mixing}, completing the derivation.

\begin{thebibliography}{99}

\bibitem{Tao2016}
T.~Tao, \emph{Finite time blow‑up for an averaged three‑dimensional Navier--Stokes equation},
J.\ Amer.\ Math.\ Soc.\ \textbf{29} (2016), 601--674.

\bibitem{GallaySmets2018}
T.~Gallay and D.~Smets, \emph{Spectral stability of inviscid vortex pairs},
Arch.\ Rational Mech.\ Anal.\ \textbf{230} (2018), 343--496.

\bibitem{Serfati1995}
P.~Serfati, \emph{Vorticité et régularité très faible : solutions appartenant à $L^\infty$ dans les équations d’Euler en dimension deux},
C.~R.\ Acad.\ Sci.\ Paris \textbf{320} (1995), 175--180.

\bibitem{Vishik1998}
M.~Vishik, \emph{Hydrodynamics in Besov spaces},
Arch.\ Rational Mech.\ Anal.\ \textbf{145} (1998), 197--214.

\bibitem{RudelsonVershynin2007}
M.~Rudelson and R.~Vershynin, \emph{Sampling from large matrices: an approach through geometric functional analysis},
J.\ ACM \textbf{54} (2007), Art.\ 21.

\bibitem{Tropp2012}
J.~A.~Tropp, \emph{User‑friendly tail bounds for sums of random matrices},
Found.\ Comput.\ Math.\ \textbf{12} (2012), 389--434.
\bibitem{Sznitman1991}
A.-S.~Sznitman, \emph{Topics in propagation of chaos}, 
in \emph{École d’Été de Probabilités de Saint‑Flour XIX—1989},
Lecture Notes in Math.\ 1464, Springer, 1991, 165–251.

\bibitem{Batchelor1953}
G.~K.~Batchelor, \emph{The Theory of Homogeneous Turbulence}, Cambridge Univ.\ Press, 1953.

\bibitem{DesjardinsGrenier2000}
B.~Desjardins and E.~Grenier, \emph{Low Mach number limit of the full Navier–Stokes equations},
Comm.\ Pure Appl.\ Math.\ \textbf{53} (2000), 1232–1270.

\bibitem{HairerMattingly2011}
M.~Hairer and J.~C.~Mattingly, \emph{A theory of hypoellipticity and unique ergodicity for semilinear stochastic PDEs},
Electron.\ J.\ Probab.\ \textbf{16} (2011), 658–738.

\bibitem{KuksinShirikyan2012}
S.~Kuksin and A.~Shirikyan, \emph{Mathematics of Two‑Dimensional Turbulence}, Cambridge Univ.\ Press, 2012.

\end{thebibliography}
\end{document}